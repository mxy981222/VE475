\documentclass[12pt,a4paper]{article}
%\usepackage{ctex}
\usepackage{amsmath,amscd,amsbsy,amssymb,latexsym,url,bm,amsthm,tabularx}
\usepackage{epsfig,graphicx,subfigure}
\usepackage{enumitem,balance}
\usepackage{wrapfig}
\usepackage{mathrsfs,euscript}
\usepackage[x11names,svgnames,dvipsnames]{xcolor}
\usepackage{hyperref}
\usepackage[vlined,ruled,commentsnumbered,linesnumbered]{algorithm2e}
\usepackage{listings}

%\usepackage{fontspec}

\newtheorem{theorem}{Theorem}
\newtheorem{lemma}[theorem]{Lemma}
\newtheorem{proposition}[theorem]{Proposition}
\newtheorem{corollary}[theorem]{Corollary}
\newtheorem{exercise}{Exercise}
\newtheorem*{solution}{Solution}
\newtheorem{definition}{Definition}
\theoremstyle{definition}

\begin{document}
\vspace*{0.25cm}

\hrulefill

\thispagestyle{empty}

\begin{center}
\begin{large}
\sc{UM--SJTU Joint Institute \vspace{0.3em} \\ Electronic Circuits \\(Ve311)}
\end{large}

\hrulefill

\vspace*{5cm}
\begin{Large}
\sc{{HW 1}}
\end{Large}
\end{center}


\vfill

\begin{table}[h!]
\flushleft
\begin{tabular}{ll}
Name: Ming Xingyu \hspace*{2em}&
ID: 517370910224\hspace*{2em}\\
\end{tabular}
\end{table}

\hfill
\newpage
\section{Ex.1}
\subsection{}


Since the text is encrypted by Caesar Cypher, we can list all the possible results. 

\noindent
GXKTG
HYLUH
IZMVI
JANWJ
KBOXK
LCPYL
MDQZM
NERAN
OFSBO
PGTCP
QHUDQ
\textit{RIVER}
SJWFS
TKXGT
ULYHU
VMZIV
WNAJW
XOBKX
YPCLY
ZQDMZ
\textit{ARENA}
BSFOB
CTGPC
DUHQD

Based on observation and my vocabulary, the text could be either RIVER or ARENA. 
\subsection{}
There are 4 letters for this word, hence, it can form a $2\times 2$ matrix. Here for dont and ELNI, we can derive
$$
\left(
\begin{matrix}
3&14\\
13&19
\end{matrix}
\right)
$$
and
$$
\left(
\begin{matrix}
4&11\\
13&8
\end{matrix}
\right)
$$

Hence, we can have the following equation
$$
\left(\begin{array}{cc}
3 & 14 \\
13 & 19
\end{array}\right)
\left(\begin{array}{ll}
a & b \\
c & d
\end{array}\right) 
\equiv
\left(\begin{array}{cc}
4 & 11 \\
13 & 8
\end{array}\right) \bmod 26
$$

$$
det\left(
\begin{matrix}
3&14\\
13&19
\end{matrix}
\right)=-125
$$

Now we need to find some $x$ such that $-125\cdot x\equiv 1\ mod\ 26$. 
We can simply have 
$$-125=-4\times 26+(-21)$$
$$26=-1\times (-21)+5$$
$$-21=-4\times 5-1$$
Hence, $1=-4\times 5+21=-4\times (26-21)+21=-4\times 26+5\times 21=4\times 26+5\times (-4\times 26+125)=(-5)\times (-125)-16\times 26$, which means $-125\times (-5)\equiv 1\ mod\ 26$ and this matrix is invertible. 

The adjacent matrix of it is 
$$
\left(
\begin{matrix}
19&-13\\
-14&3
\end{matrix}
\right)
$$

And the inverse of it is 
$$
\left(
\begin{matrix}
-95&70\\
65&-15
\end{matrix}
\right)
$$

Now, we can calculate the key by matrix multiplication as follows, 
$$
\left(\begin{array}{ll}
a & b \\
c & d
\end{array}\right) 
\equiv
\left(
\begin{matrix}
-95&70\\
65&-15
\end{matrix}
\right)\times 
\left(\begin{array}{cc}
4 & 11 \\
13 & 8
\end{array}\right) \bmod 26
$$
$$
\left(\begin{array}{ll}
a & b \\
c & d
\end{array}\right) 
\equiv
\left(\begin{array}{cc}
530 & -485 \\
65 & 595
\end{array}\right) \bmod 26
$$

Finally, we have 
$$
\left(\begin{array}{ll}
a & b \\
c & d
\end{array}\right) 
=
\left(\begin{array}{cc}
10 & 9 \\
13 & 23
\end{array}\right)
$$
\subsection{}
Since, $n\mid ab$, we can simply derive $ab=kn,\ k\in \mathbb{N}$. 

\noindent Then $b=\frac{kn}{a}$. Because $gcd(a,n)=1\land b\in \mathbb{N}$, we have $a\mid k$, 


\noindent i.e. $k=ca,\ c\in \mathbb{N}$. 

\noindent Hence, $b=cn\Leftrightarrow n\mid b$. 
\subsection{}
By Chinese Reminder Theorem, we can have 
$$30030=116\times 257+218$$
$$257=218+39$$
$$218=39\times 5+23$$
$$39=23+16$$
$$23=16+7$$
$$16=7\times 2 +2$$
$$7=2\times 3+1$$
$$2=2\times 1$$
Hence, $gcd(30030,257)=1$. 

Noticing that $\sqrt[2]{257}=16.03122$, hence, the prime factor of it can only be among 2, 3, 5, 7, 11, 13. By brutal force we can have 
$$257=128\times 2+1$$
$$257=85\times 3+2$$
$$257=51\times 5+2$$
$$257=36\times 7+5$$
$$257=23\times 11+4$$
$$257=19\times 13+8$$
So, there is no factor for 257, which indicates 257 is prime. 
\subsection{}
We learn from the lecture that OTP use bitwise-XOR operation on the plaintext and the key. 

For the XOR operation we can have $a\oplus b=c\Rightarrow a\oplus c=b$. So, for the CPCA, the enemy can derive the key and when we reuse it, the cipertext can be easily decipered. 
\subsection{}
According to the lecture slides, we know that in order to make it safe, it must let the attacker to do at least $2^128$ times computation, hence, 
$$\sqrt{nlogn}\geq128$$
$$nlogn\geq 16384$$
For the algorithm implementation, the base number may varies. Here, we assume the base number to be 2. And the solution is 
$$n\geq 1546.43$$

Since $n\in \mathbb{N}$, the size should be larger than 1547. 
\newpage
\section{}
\subsection{}
Firstly, for the Vigenère cipher, it has a Table, as shown below
\begin{figure}[h!]
\centering
\includegraphics[scale=0.5]{vc.png}
\caption{Vigenère cipher table. }
\end{figure}


Secondly, similar to the Caesar cipher, the principle of Vigenère cipher is to shift the letter according to the table by the key. 

Then, the key is a key word, in which every letter indicate the line we look-up in the table, which will repeat itself until the length of it is the same as the plain-text during encryption. 

During encryption, one letter in the key corresponding to one letter in the plain-text. For example, we have $fat$ for the plain-text and $meat$ for the key, we have 
$$f\rightarrow m\Rightarrow R$$
$$a\rightarrow e\Rightarrow E$$
$$t\rightarrow a\Rightarrow T$$
And the ciphertext is RET. 
\subsection{}
\subsubsection{}
Since the cipher-text may repeat itself every six letters, it is reasonable for Eve to suspect the plain-text is some repeated letters with length as 1, 2, 3, 6(dividers of 6). 
\subsubsection{}
As mentioned above, the cipher-text repeated every 6 letters. The length of the key is very likely to be 6. 
\subsubsection{}
According to the helpful hint, we know that there is no English word of length 6 is a shift of another English word. If Eve guess the plain-text to be a repeated letter, she can simply look-up the dictionary to find all the English words and find which one matches the encryption. 
\end{document}